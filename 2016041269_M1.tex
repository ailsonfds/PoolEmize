\documentclass[a4paper,12pt]{article}
\usepackage[utf8]{inputenc}
\usepackage{ae}
\usepackage{pslatex}
\usepackage[portuges]{babel}
\usepackage{indentfirst}
\usepackage[portuguese,noprefix]{nomencl}
\usepackage{setspace}
\usepackage[colorlinks,allcolors=black]{hyperref}
\usepackage{enumitem}
\usepackage[ampersand]{easylist}
\usepackage{fancyhdr}

\setlength{\oddsidemargin}{2.5cm}
\setlength{\evensidemargin}{2.5cm}
\setlength{\textwidth}{15cm}
\addtolength{\oddsidemargin}{-1in}
\addtolength{\evensidemargin}{-1in}
\setlength{\topmargin}{2.0cm}
\setlength{\headheight}{1.0cm}
\setlength{\headsep}{1.0cm}
\setlength{\textheight}{22.7cm}
\setlength{\footskip}{1.0cm}
\addtolength{\topmargin}{-1in}
\singlespacing

\begin{document}
\begin{titlepage}
\begin{center}
\centering
\linespread{2.0}
\Huge\Huge\textbf{M1} \\
\huge\huge\textbf{Projeto de modelagem} \\
\LARGE\LARGE\textbf{Sistema de gerenciamento de votos} \\
\vfill
\end{center}
\textbf{Discente: } Ailson Forte dos Santos \\
\textbf{Disciplina: } DIM0504 - Análise de Projeto Orientado a Objetos (APOO) \\
{\par\hfill\today}
\end{titlepage}
\newpage
\tableofcontents

\newpage
\section*{Descrição da aplicação}
\markright{}
\addcontentsline{toc}{section}{Descrição da aplicação}
Desenvolveremos nesse projeto um sitema que visa gerenciar enquetes e elições. Seja para propósitos simples como a cor favorita de determinado grupo de indivíduos ou algo mais complexo como a eleição para presidente de um país.
\par
Essa aplicação tem como intuíto facilitar o gerenciamento de pesquisas com enfoques estatísticos. Nela, o usuário poderá criar e manipular sua própria pesquisa ou eleição.

\newpage
\section*{Requisitos}
\markright{}
\addcontentsline{toc}{section}{Requisitos}
\subsection*{Regras gerais deste sistema:}
\markright{}
\addcontentsline{toc}{subsection}{Regras gerais deste sistema}

\begin{easylist}[itemize]
& O sistema possui eleitores e juizes de eleição.
& As eleições só devem ser cadastradas por juízes de eleição.
& Só quem pode votar são os eleitores e juízes de eleição.
& Juízes de eleição não podem modificar votos, apenas validá-los.
& Todo voto é secreto e direto, ou seja, juízes não podem ver quem votou mas devem saber o resultado da votação.
\end{easylist}

\end{document}

http://www.ebc.com.br/noticias/politica/2013/07/como-funciona-o-sistema-eleitoral-brasileiro
https://pt.wikipedia.org/wiki/Sistema_de_vota%C3%A7%C3%A3o
